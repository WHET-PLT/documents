\documentclass[letterpaper]{article}
\usepackage{fullpage, amsmath, amssymb, enumerate, tabularx, listings, color, float}
%take up ALLOFTHEPAGE
\usepackage[margin=0.75in]{geometry}
%for footnotes
\usepackage[bottom]{footmisc}

%for the code example parts
\definecolor{gray}{gray}{0.5}
\lstset{numbers=left, numbersep=5pt, numberstyle=\tiny \color{gray}, frame=l, commentstyle=\color{gray}, comment=[l]{\/\/}}

%document details
\author{William Falk-Wallace (wgf2104), Hila Gutfreund (hg2287), \\Emily Lemonier (eql2001), Thomas Elling (tee2103)}
\title{COMSW4115: Programming Languages and Translators\\The DJ Language Reference Manual}
\date{\today}

\begin{document}

%first page: title and toc
\maketitle
\tableofcontents 
\pagebreak[4]

%INTRODUCTION
\section{Introduction}
We propose a procedural scripting language, DJ, which provides a programming paradigm for algorithmic music production. Through its utilization of themes and motifs, music is naturally repetitive and often dynamic. DJ provides control-flow mechanisms, including \texttt{for} and \texttt{loop} functions, which simplify the development of structured iterative music. The DJ Language also makes use of conditional logic and offers built-in effects (including pitch bend, tremolo and vibrato).
%Moreover, it supports extensible sound banks to facilitate the production of deeply textured musical compositions.
Our goal in the specification of The DJ Language is to abstract away the intricacies and limitations of the MIDI specification, including channeling, patch-maps and instrumentation, allowing the artist to focus on her or his work: composing music.

%LEXICAL CONVENTIONS
%|_COMMENTS
%|_IDENTIFIERS
%|_KEYWORDS
%|_CONSTANTS AND STRUCTURES
%|_SEPARATORS
%|_WHITE SPACE
\section{Lexical Conventions}

%COMMENTS
\subsection{Comments} 
Comments are initialized by the character sequence \texttt{/*} and terminated by the first following character sequence \texttt{*/}.

%IDENTIFIERS
\subsection{Identifiers}
An identifier is a sequence of letters, underscores and digits; note that in identifiers, uppercase and lowercase letters correspond to different characters. The first character of an identifier is a letter [`a'-`z'] or [`A' - `Z'].

%KEYWORDS
\begin{samepage}
\subsection{Keywords}
Keywords are reserved identifiers and may not be redefined. They are used for control structure, constants, as well as system level function calls.
\begin{table}[H]
\centering
\begin{tabularx}{.75\textwidth}{|X|X|X|}
\hline
\texttt{int} & \texttt{note} & \texttt{rest} \\
\hline
\texttt{chord} & \texttt{track} & \texttt{song} \\
\hline
\texttt{array} & \texttt{if} & \texttt{else} \\
\hline
\texttt{for} & \texttt{return} & \texttt{loop} \\
\hline
\texttt{fun} & \texttt{vol} & \texttt{dur} \\
\hline
\end{tabularx}
\end{table}
\end{samepage}

%CONSTANTS AND STRUCTURES
\subsection{Constants and Structures}
\subsubsection{Integers}
An integer constant is a primitive data type which represents some finite subset of the mathematical integers. If `-' is prepended to the integer, the value of the integer is considered negative (ex: \texttt{int x = -22}). An integer may take a value between $-2^{30}$ and $2^{30}-1$ on 32-bit systems and up to $-2^{62}$ to $2^{62}-1$. %These ranges are a reflection of the OCAML standards.
%found those numbers on the ocaml site
\subsubsection{Note}
Note literals are the most basic units of a song, and are represented using the following notation: (pitch, instrument, volume, duration).
\subsubsection{Rest}
A rest literal is a basic unit of a composition (and DJ program) that doesn't have a pitch, instrument, or volume, but does maintain a duration.
%\subsection{Data Types}
\subsubsection{Array}
Arrays ummm... TODO 
\subsubsection{Chord}
A primitive data structure representing a collection of Notes which are intended to be performed beginning in the same beat. 
\subsubsection{Track}
A collection of Chords which follow linearly and are all intended to be played by the same instrument.

%SEPARATORS
\subsection{Separators}
A separator distinguishes tokens. White space is a separator and is discussed in the next section, but it is not a token. All other separators are single-character tokens. These separators include \texttt{( ) \textless\  \textgreater\  ;}. Note that \texttt{;}  is used to indicate the end of an expression or statement. 

%WHITESPACE
\subsection{White Space}
White space collectively refers to the space character, the tab character, and the newline character. White space is used to separate tokens and is otherwise ignored. Wherever one space is allowed, any length of white space is allowed.
For example: $y$$=$$x$$+$$5$ is equivalent to $y$\textvisiblespace $=$\textvisiblespace  \textvisiblespace $x$\textvisiblespace $+$\textvisiblespace $5$, since $y$, $x$, $5$, $+$, and $=$ are all complete tokens.

%EXPRESSIONS AND OPERATORS
\section{Expressions and Operators}
An operator is a special token that specifies an action performed on either one or two operands. Operator precedence is specified in the order of appearance in the following sections of this document; directional associativity is also specified for each operator. The order of evaluation of all other expressions is left to the compiler and is not guaranteed.

%DECLARATIONS
\subsection{Variable Declaration}
Declarations dictate the class type of identifiers. Declarations take the form
\texttt{type-specifier identifier}, and are optionally followed by declarators of the form \texttt{type-specifier (expression)}.

%PRIMARY EXPRESSIONS
\subsection{Fundamental Expressions}
Fundamental expressions consist of function calls and those expressions accessed using \texttt{-\textgreater} (described below); these are grouped rightwardly.
%It's a word. I checked.


%===STICK WITH DOT FOR CONCAT, USE -\textgreater FOR attribute/internal operator/method ACCESS===
%#BECAUSEITSPRETTY

%this is a bit close to the C LRM, so let's play around with it
\subsubsection{Identifiers} 
Identifiers are primary expressions whose types and values are specified in their declarations.
\subsubsection{Constants}
Integer, note, rest, array, chord, and track constants are primary expressions.
\subsubsection{(expression)}
A parenthesized expression is a primary expression and is in all ways equivalent to the non-parenthesized expression.
\subsubsection{primary [expr]}
An expression in square brackets following a primary expression is a primary expression. It specifies array element, note attribute, or chord or track member addressing. 
\subsubsection{primary (args...)}
An parenthesized expression following a primary expression is a primary expression. It specifies a function call which may accept a variable-length, comma-separated list of parameters \texttt{args}.
\subsubsection{primary \texttt{-\textgreater} attribute}
A primary expression followed by \texttt{-\textgreater} and an attribute name is a primary expression. It specifies primitive data type structure-attribute access.
\subsection{Unary Operators}
Unary Operators are right-to-left associative.
\subsubsection{$-$ expression}
If the expression resolves to an integer data-type, the `-' operator causes the expression to be considered as a negative value.
\subsubsection{expression $++$}
This expression behaves as a shorthand for taking the expression result and depending on its type, incrementing its value: for integer types this means an incremental increase in value; for notes it increases pitch by one tonal half step (whole integer increase); and for chords and tracks, it increments all member-note pitches. 
\subsubsection{expression $--$}
This expression behaves as above, decrementing instead of incrementing.
\subsection{Effects}
\subsubsection{expression\textasciicircum}
This expression takes the notes in the left operand and creates a vibrator effect on each individual note. 
\subsubsection{expression\textasciitilde}
This expression takes the notes in the left operand and creates a tremelo effect on each individual note.
\subsubsection{expression \% expression}
Pitch bend


\subsection{Multiplicative Operators}
Multiplicative operators are left-to-right associative.
\subsubsection{expression $*$ expression}
\subsubsection{expression $/$ expression}
Integer multiplication and division act as expected, except the division operator, $/$, truncates its result to the nearest integer; it computes $\lfloor{expr / expr}\rfloor$

\subsection{Additive Operators}
Additive operators are left-to-right associative.
\subsubsection{expression $+$ expression}
operators must be notes, chords, tracks, or ints and the result is pitch addition for ... and integer arithmetic for ints.

This expression takes the notes or chords specifed in the left operand and increases the notes or the individual notes in the chord by the number of half steps specified by the right operand. 

\subsubsection{expression $-$ expression}
This expression behaves like previous expression except the notes in the left operand are decreased by the specified number of half steps specified by the right operand. 
\subsubsection{expression $.$ expression}
This expression takes the tracks in the right operand and concatonates them to the first track on the left operand. A third new track is returned containing the concatenated tracks. Notes are elevated to size one Chords and Chords are elevated to Tracks before concatenation. 
\subsubsection{expression $:$ expression}
This expression takes the notes, chords, or tracks on the right hand side and parallel adds them to the current note, chord, or track. When used on Notes it returns a new Chord containing both Notes; when used on Chords it returns a new Chord representing the union of the original Chords; when used on tracks it returns a new Track such that the Chords are added in parallel by corresponding time tick, with no added offset. 


\subsection{Relational Operators}
Left to right
\subsubsection{expression $<$ expression}
This expression checks whether all notes within the left operand are less than all the notes within the right operand
\subsubsection{expression $>$ expression}
This expression checks whether all notes within the left operand are greater than all the notes within the right operand
\subsubsection{expression $<=$ expression}
This expression checks whether all notes within the left operand are less than or equal to all the notes within the right operand
\subsubsection{expression $>=$ expression}
This expression checks whether all notes within the left operand are greater than or equal to all the notes within the right operand
\subsubsection{expression $==$ expression}
This expression checks whether all notes within two operands are equal to one another. 
\subsubsection{expression $!=$ expression}
This expression checks whether all notes within two operands are not equal to one another

\subsection{Assignment Operators}
right to left
\subsubsection{lvalue $=$ expression}
LVALUES?
evaluates to...
\subsection{Declarations}
type-specifier(args...)\\
eg. note(5, 3, 1...); track(); fun
function definitions here??


\section{Statements}
Statements cause actions and are responsible for control flow within your programs.
\subsection{Expression Statement}
Any statement can turn into an expression by adding a semicolon to the end of the expression (ex: 2+2;). 
\subsection{The \texttt{if} Statement}
We use the \texttt{if} statement to conditionally execute part of a program, based on the truth value of a given expression.
General form of if statement: \\
if (test) \\
then-statement \\
else
else-statement (make this safe format as the examples?)
\subsection{The \texttt{for} Statement}
\subsection{The \texttt{return} Statement}
Causes the current function call to end in the current sub-routine and return to where the function was called. The return function can return nothing (\texttt{return;}) or a return value can be passed back to the calling function (\texttt{return expression;}).
\section{Functions}
\subsection{Defining Functions}
Functions are defined by a function name followed by parenthesis that contain parameters to the function separated by commas. All functions must have a return statement. The function body is contained between a curly brace at the beginning and a curly brace at the end of the function.

mergeTrack (track1, track2) \{ \\
		/*stuff*/ \\
	return newtrack;\\
\}


\subsection{The \texttt{song} Function}
The \texttt{song} function is where the tracks a user has created will be modified and/or combined. This is where the music is essentially created. The \texttt{song} function returns an array of tracks which represent the complete song.

\begin{samepage}
\subsection{Reserved Functions}
\begin{table}[H]
\centering
SOME NOT FUNCTIONS, JUST ATTRIBUTES; like vol/dur...
\begin{tabularx}{.75\textwidth}{|c|X|}
\hline
print(expression) & print to console \\
\hline
loop(integer) & Loops a given Note, Chord, or Track the over number of beats specified. If given a number of beats fewer than the total track size (n.b. implicit elevation occurs as necessary), first $<$int$>$ beats will be included. \\
\hline
repeat(integer) & Repeats a given Note, Chord, or Track $<$int$>$ times, returning a new Track. \\
\hline
add(integer) & Adds a Chord to a Track. \\
\hline
strip(integer) & Removes all instances of Chord from a Track. \\
\hline
remove(integer) & Removes Chord from Track at designated location. \\
\hline
\end{tabularx}
\end{table}
\end{samepage}

\subsection{Function/Variable Scoping}
Braces determine the scope of a function/variables. For example, if a variable is declared within a function, it is a local variable to that function and can only be accessed in that function. That local variable would be defined within the braces of a function body. A global variable would be defined outside the scope of braces.

\section{Compile Process and Output Files}
CSV to MIDI to Java. CSV2MIDI Java Class. 
%http://code.google.com/p/midilc/source/browse/trunk/src/components/CSV2MIDI.java?r=122







%=========================
%=========================
%=========================
\section{NOTES TO BE DISPERSED INTO SECTIONS ABOVE}

\begin{itemize}
\item
Note, Chord, and Track are defined as primitives and are hierarchical. The hierarchy is as follows: Tracks are composed of Chords, which are composed of Notes and Rests.
\item
Notes are represented by ordered seven-tuples defining characteristic attributes, including pitch, instrumentation, volume, duration (in beats), the presence of effects including tremolo, vibrato, and pitch bend. The primitive Rest object allows for a pause in a Track.
\item
Tracks, Chords, and Notes may be added in series or parallel. A new Track is produced by adding Tracks in series or parallel. Chords produce Tracks when added in series. Notes added produce Chords when added in parallel.
\item
Several mutative operators exist for manipulating Note attributes at the Note, Chord, and Track level.
\item
All programs consist of a single main function, called \texttt{SONG}, that returns an array of tracks, intended to start simultaneously and be played in parallel. Each array element can be considered as a polyphonic MIDI channel. This array of tracks is compiled into a bytecode file containing the complete set of MIDI-messages required to produce the programmed song. A third party bytecode-to-MIDI interpreter will be used to produce the final sound file.
\item
Song-wide properties are specified to the compiler. Attributes such as tempo/beats per minute and channel looping are available as compiler options.
\item
This structure, as well as the use of the MIDI specification and interface, allows for a fairly extensible language and production capability. For example, through the manipulation or linking of sound banks, new sounds and samples are able to be incorporated to produce rich and interesting programmatic music.
\end{itemize}

\section{Syntax}

The following subsections and tables represent the primitives, operators, and functions defined in the DJ Language specification.

\begin{samepage}
\subsection{Primitives}
\begin{table}[H]
\centering
\begin{tabularx}{.75\textwidth}{|c|X|}
\hline
Integer & Used for addressing and specifying Note/Chord/Track attributes. \\
\hline
Array & Fixed-length collection of elements (int, Note, Chord, Track), each identified by at least one array index. \\
\hline
Note & Ordered tuple containing pitch (pitch), instrument (instr), volume (vol), duration (dur), tremolo (trem), vibrato (vib), pitch bend (pb) (n.b. pitch number is sequentially numbered in tonal half-step increments; tremolo and vibrato attributes are boolean). \\
\hline
Rest & A durational note with no volume and no pitch and which is not responsive to pitch, volume, or effect operations. \\
\hline
Chord & Vector of Notes (size $\geq$ 1). \\
\hline
Track & Vector of Chords (size $\geq$ 1). \\
\hline
\end{tabularx}
\end{table}
\end{samepage}

\begin{samepage}
\subsection{Operators}
\begin{table}[H]
\centering
\begin{tabularx}{.75\textwidth}{|c|X|}
\hline
$>$, $<$ & Pitchbend: changes the pitch bend of a Note, the Notes of a Chord, or all Notes of a Track. (binary) \\
\hline
$+$, $-$ & Increase/Decrease pitch of an individual note, all Notes in a Chord, or all Notes in a Track, respectively, by a specified amount. (binary) \\
\hline
$++$, $--$ & Increase/Decrease respective pitch of Notes, either atomically or in a Chord or Track by a single integer increment (tonal half-step). (unary) \\
\hline
$[<$int$>]$ & Address Array, Chord, or Track element at given index. (unary) \\
\hline
$\sim$ & Creates a tremelo effect on the individual note, all Notes in the Chord, or all Notes in the Track that it operates on. (unary) \\
\hline
$\wedge$ & Creates a vibratro effect on the individual note, all Notes in the Chord, or all Notes in the Track that it operates on. (unary) \\
\hline
: & Parallel Add: adds Notes, Chords, or Tracks in parallel. When used on Notes, returns a new Chord containing both Notes; when used on Chords, returns a new Chord representing the union of both original Chords; when used with Tracks, returns a new Track such that Chords are added in parallel by corresponding time tick, with no added offset. (binary) \\
\hline
. & Serial Add: both operands must be Tracks. The right operand is concatenated to the first, and a third, new  Track is returned. Notes are elevated to size-one Chords and Chords are elevated to Tracks before concatenating. (binary) \\
\hline
$=$ & Assignment operator. (binary) \\
\hline
$+=$ & Integer Add-in-place. (binary)\\
\hline
$|$ & Conditional OR. (binary)\\
\hline
\& & Conditional AND. (binary)\\
\hline
$==$ & Logical equality (deep). (binary) \\
\hline
\end{tabularx}
\end{table}
\end{samepage}

\begin{samepage}
\subsection{Functions}
\begin{table}[H]
\centering
\begin{tabularx}{.75\textwidth}{|c|X|}
\hline
vol($<$int$>$) & Change Chord/Note/Track volume (integer value 0-99). (absolute) \\
\hline
dur($<$int$>$) & Change Chord/Note duration (number of beats). (absolute) \\
\hline
loop($<$int$>$) & Loops a given Note, Chord, or Track the over number of beats specified. If given a number of beats fewer than the total track size (n.b. implicit elevation occurs as necessary), first $<$int$>$ beats will be included. \\
\hline
repeat($<$int$>$) & Repeats a given Note, Chord, or Track $<$int$>$ times, returning a new Track. \\
\hline
add($<$chord$>$) & Adds a Chord to a Track. \\
\hline
strip($<$chord$>$) & Removes all instances of Chord from a Track. \\
\hline
remove($<$int$>$) & Removes Chord from Track at designated location. \\
\hline
\end{tabularx}
\end{table}
\end{samepage}

\begin{samepage}
\subsection{Reserved Words and Conditionals}
\begin{table}[H]
\centering
\begin{tabularx}{.75\textwidth}{|c|X|}
\hline
if (\textit{expr}) \{...\} else \{...\}& Paired control flow statement that acts upon the logical expression within the \texttt{if} statement parentheses. If the expression evaluates to true, the control flow will continue to the code contained within the braces of the \texttt{if} body. If the argument is false, then control flow moves on to the code in the braces of the \texttt{else} body. \\
\hline
return & Terminates control flow of the current function and returns control flow to the calling function, passing immediately subsequent primitive to calling function.\\
\hline
null & Undefined object identifier; used in declaring non-\texttt{return}ing functions. \\
\hline
int, Array Note, Rest, Chord, Track & Type declaration specifiers.\\
\hline
SONG \{\} & Conventional "main" function declaration, with unspecified return type, which indicates program outset to the compiler.\\
\hline
\end{tabularx}
\end{table}
\end{samepage}

\section{Examples}

\begin{minipage}{\linewidth}
\subsection{Example 1: Arpeggio}
\lstinputlisting{examples/arpeggio_dj.txt}
\end{minipage}

\begin{minipage}{\linewidth}
\subsection{Example 2: Loop With Effects}
\lstinputlisting{examples/loop_with_effects_dj.txt}
\end{minipage}

\begin{minipage}{\linewidth}
\subsection{Example 3: Add/Remove Notes \& Chords}
\lstinputlisting{examples/addRemoveChords_dj.txt}
\end{minipage}

\end{document}




